\documentclass[10pt,a4paper,onecolumn]{article}
\usepackage{marginnote}
\usepackage{graphicx}
\usepackage{xcolor}
\usepackage{authblk,etoolbox}
\usepackage{titlesec}
\usepackage{calc}
\usepackage{tikz}
\usepackage{hyperref}
\hypersetup{colorlinks,breaklinks=true,
            urlcolor=[rgb]{0.0, 0.5, 1.0},
            linkcolor=[rgb]{0.0, 0.5, 1.0}}
\usepackage{caption}
\usepackage{tcolorbox}
\usepackage{amssymb,amsmath}
\usepackage{ifxetex,ifluatex}
\usepackage{seqsplit}
\usepackage{xstring}

\usepackage{float}
\let\origfigure\figure
\let\endorigfigure\endfigure
\renewenvironment{figure}[1][2] {
    \expandafter\origfigure\expandafter[H]
} {
    \endorigfigure
}


\usepackage{fixltx2e} % provides \textsubscript
\usepackage[
  backend=biber,
%  style=alphabetic,
%  citestyle=numeric
]{biblatex}
\bibliography{paper.bib}

% --- Splitting \texttt --------------------------------------------------

\let\textttOrig=\texttt
\def\texttt#1{\expandafter\textttOrig{\seqsplit{#1}}}
\renewcommand{\seqinsert}{\ifmmode
  \allowbreak
  \else\penalty6000\hspace{0pt plus 0.02em}\fi}


% --- Pandoc does not distinguish between links like [foo](bar) and
% --- [foo](foo) -- a simplistic Markdown model.  However, this is
% --- wrong:  in links like [foo](foo) the text is the url, and must
% --- be split correspondingly.
% --- Here we detect links \href{foo}{foo}, and also links starting
% --- with https://doi.org, and use path-like splitting (but not
% --- escaping!) with these links.
% --- Another vile thing pandoc does is the different escaping of
% --- foo and bar.  This may confound our detection.
% --- This problem we do not try to solve at present, with the exception
% --- of doi-like urls, which we detect correctly.


\makeatletter
\let\href@Orig=\href
\def\href@Urllike#1#2{\href@Orig{#1}{\begingroup
    \def\Url@String{#2}\Url@FormatString
    \endgroup}}
\def\href@Notdoi#1#2{\def\tempa{#1}\def\tempb{#2}%
  \ifx\tempa\tempb\relax\href@Urllike{#1}{#2}\else
  \href@Orig{#1}{#2}\fi}
\def\href#1#2{%
  \IfBeginWith{#1}{https://doi.org}%
  {\href@Urllike{#1}{#2}}{\href@Notdoi{#1}{#2}}}
\makeatother

\newlength{\cslhangindent}
\setlength{\cslhangindent}{1.5em}
\newlength{\csllabelwidth}
\setlength{\csllabelwidth}{3em}
\newenvironment{CSLReferences}[3] % #1 hanging-ident, #2 entry spacing
 {% don't indent paragraphs
  \setlength{\parindent}{0pt}
  % turn on hanging indent if param 1 is 1
  \ifodd #1 \everypar{\setlength{\hangindent}{\cslhangindent}}\ignorespaces\fi
  % set entry spacing
  \ifnum #2 > 0
  \setlength{\parskip}{#2\baselineskip}
  \fi
 }%
 {}
\usepackage{calc}
\newcommand{\CSLBlock}[1]{#1\hfill\break}
\newcommand{\CSLLeftMargin}[1]{\parbox[t]{\csllabelwidth}{#1}}
\newcommand{\CSLRightInline}[1]{\parbox[t]{\linewidth - \csllabelwidth}{#1}}
\newcommand{\CSLIndent}[1]{\hspace{\cslhangindent}#1}

% --- Page layout -------------------------------------------------------------
\usepackage[top=3.5cm, bottom=3cm, right=1.5cm, left=1.0cm,
            headheight=2.2cm, reversemp, includemp, marginparwidth=4.5cm]{geometry}

% --- Default font ------------------------------------------------------------
\renewcommand\familydefault{\sfdefault}

% --- Style -------------------------------------------------------------------
\renewcommand{\bibfont}{\small \sffamily}
\renewcommand{\captionfont}{\small\sffamily}
\renewcommand{\captionlabelfont}{\bfseries}

% --- Section/SubSection/SubSubSection ----------------------------------------
\titleformat{\section}
  {\normalfont\sffamily\Large\bfseries}
  {}{0pt}{}
\titleformat{\subsection}
  {\normalfont\sffamily\large\bfseries}
  {}{0pt}{}
\titleformat{\subsubsection}
  {\normalfont\sffamily\bfseries}
  {}{0pt}{}
\titleformat*{\paragraph}
  {\sffamily\normalsize}


% --- Header / Footer ---------------------------------------------------------
\usepackage{fancyhdr}
\pagestyle{fancy}
\fancyhf{}
%\renewcommand{\headrulewidth}{0.50pt}
\renewcommand{\headrulewidth}{0pt}
\fancyhead[L]{\hspace{-0.75cm}\includegraphics[width=5.5cm]{/usr/local/share/openjournals/joss/logo.png}}
\fancyhead[C]{}
\fancyhead[R]{}
\renewcommand{\footrulewidth}{0.25pt}

\fancyfoot[L]{\parbox[t]{0.98\headwidth}{\footnotesize{\sffamily ¿citation\_author?, (2024). omni-fig:
Unleashing Project Configuration and Organization in
Python. \textit{Journal of Open Source Software}, ¿VOL?(¿ISSUE?), ¿PAGE?. \url{https://doi.org/DOI unavailable}}}}


\fancyfoot[R]{\sffamily \thepage}
\makeatletter
\let\ps@plain\ps@fancy
\fancyheadoffset[L]{4.5cm}
\fancyfootoffset[L]{4.5cm}

% --- Macros ---------

\definecolor{linky}{rgb}{0.0, 0.5, 1.0}

\newtcolorbox{repobox}
   {colback=red, colframe=red!75!black,
     boxrule=0.5pt, arc=2pt, left=6pt, right=6pt, top=3pt, bottom=3pt}

\newcommand{\ExternalLink}{%
   \tikz[x=1.2ex, y=1.2ex, baseline=-0.05ex]{%
       \begin{scope}[x=1ex, y=1ex]
           \clip (-0.1,-0.1)
               --++ (-0, 1.2)
               --++ (0.6, 0)
               --++ (0, -0.6)
               --++ (0.6, 0)
               --++ (0, -1);
           \path[draw,
               line width = 0.5,
               rounded corners=0.5]
               (0,0) rectangle (1,1);
       \end{scope}
       \path[draw, line width = 0.5] (0.5, 0.5)
           -- (1, 1);
       \path[draw, line width = 0.5] (0.6, 1)
           -- (1, 1) -- (1, 0.6);
       }
   }

% --- Title / Authors ---------------------------------------------------------
% patch \maketitle so that it doesn't center
\patchcmd{\@maketitle}{center}{flushleft}{}{}
\patchcmd{\@maketitle}{center}{flushleft}{}{}
% patch \maketitle so that the font size for the title is normal
\patchcmd{\@maketitle}{\LARGE}{\LARGE\sffamily}{}{}
% patch the patch by authblk so that the author block is flush left
\def\maketitle{{%
  \renewenvironment{tabular}[2][]
    {\begin{flushleft}}
    {\end{flushleft}}
  \AB@maketitle}}
\makeatletter
\renewcommand\AB@affilsepx{ \protect\Affilfont}
%\renewcommand\AB@affilnote[1]{{\bfseries #1}\hspace{2pt}}
\renewcommand\AB@affilnote[1]{{\bfseries #1}\hspace{3pt}}
\renewcommand{\affil}[2][]%
   {\newaffiltrue\let\AB@blk@and\AB@pand
      \if\relax#1\relax\def\AB@note{\AB@thenote}\else\def\AB@note{#1}%
        \setcounter{Maxaffil}{0}\fi
        \begingroup
        \let\href=\href@Orig
        \let\texttt=\textttOrig
        \let\protect\@unexpandable@protect
        \def\thanks{\protect\thanks}\def\footnote{\protect\footnote}%
        \@temptokena=\expandafter{\AB@authors}%
        {\def\\{\protect\\\protect\Affilfont}\xdef\AB@temp{#2}}%
         \xdef\AB@authors{\the\@temptokena\AB@las\AB@au@str
         \protect\\[\affilsep]\protect\Affilfont\AB@temp}%
         \gdef\AB@las{}\gdef\AB@au@str{}%
        {\def\\{, \ignorespaces}\xdef\AB@temp{#2}}%
        \@temptokena=\expandafter{\AB@affillist}%
        \xdef\AB@affillist{\the\@temptokena \AB@affilsep
          \AB@affilnote{\AB@note}\protect\Affilfont\AB@temp}%
      \endgroup
       \let\AB@affilsep\AB@affilsepx
}
\makeatother
\renewcommand\Authfont{\sffamily\bfseries}
\renewcommand\Affilfont{\sffamily\small\mdseries}
\setlength{\affilsep}{1em}


\ifnum 0\ifxetex 1\fi\ifluatex 1\fi=0 % if pdftex
  \usepackage[T1]{fontenc}
  \usepackage[utf8]{inputenc}

\else % if luatex or xelatex
  \ifxetex
    \usepackage{mathspec}
    \usepackage{fontspec}

  \else
    \usepackage{fontspec}
  \fi
  \defaultfontfeatures{Ligatures=TeX,Scale=MatchLowercase}

\fi
% use upquote if available, for straight quotes in verbatim environments
\IfFileExists{upquote.sty}{\usepackage{upquote}}{}
% use microtype if available
\IfFileExists{microtype.sty}{%
\usepackage{microtype}
\UseMicrotypeSet[protrusion]{basicmath} % disable protrusion for tt fonts
}{}

\usepackage{hyperref}
\hypersetup{unicode=true,
            pdftitle={omni-fig: Unleashing Project Configuration and Organization in Python},
            pdfborder={0 0 0},
            breaklinks=true}
\urlstyle{same}  % don't use monospace font for urls

% --- We redefined \texttt, but in sections and captions we want the
% --- old definition
\let\addcontentslineOrig=\addcontentsline
\def\addcontentsline#1#2#3{\bgroup
  \let\texttt=\textttOrig\addcontentslineOrig{#1}{#2}{#3}\egroup}
\let\markbothOrig\markboth
\def\markboth#1#2{\bgroup
  \let\texttt=\textttOrig\markbothOrig{#1}{#2}\egroup}
\let\markrightOrig\markright
\def\markright#1{\bgroup
  \let\texttt=\textttOrig\markrightOrig{#1}\egroup}


\IfFileExists{parskip.sty}{%
\usepackage{parskip}
}{% else
\setlength{\parindent}{0pt}
\setlength{\parskip}{6pt plus 2pt minus 1pt}
}
\setlength{\emergencystretch}{3em}  % prevent overfull lines
\providecommand{\tightlist}{%
  \setlength{\itemsep}{0pt}\setlength{\parskip}{0pt}}
\setcounter{secnumdepth}{0}
% Redefines (sub)paragraphs to behave more like sections
\ifx\paragraph\undefined\else
\let\oldparagraph\paragraph
\renewcommand{\paragraph}[1]{\oldparagraph{#1}\mbox{}}
\fi
\ifx\subparagraph\undefined\else
\let\oldsubparagraph\subparagraph
\renewcommand{\subparagraph}[1]{\oldsubparagraph{#1}\mbox{}}
\fi

\title{omni-fig: Unleashing Project Configuration and Organization in
Python}

        \author[1]{Felix Leeb}
    
      \affil[1]{Max Planck Institute for Intelligent Systems, Tübingen,
Germany}
  \date{\vspace{-7ex}}

\begin{document}
\maketitle

\marginpar{

  \begin{flushleft}
  %\hrule
  \sffamily\small

  {\bfseries DOI:} \href{https://doi.org/DOI unavailable}{\color{linky}{DOI unavailable}}

  \vspace{2mm}

  {\bfseries Software}
  \begin{itemize}
    \setlength\itemsep{0em}
    \item \href{N/A}{\color{linky}{Review}} \ExternalLink
    \item \href{NO_REPOSITORY}{\color{linky}{Repository}} \ExternalLink
    \item \href{DOI unavailable}{\color{linky}{Archive}} \ExternalLink
  \end{itemize}

  \vspace{2mm}

  \par\noindent\hrulefill\par

  \vspace{2mm}

  {\bfseries Editor:} \href{https://example.com}{Pending
Editor} \ExternalLink \\
  \vspace{1mm}
    {\bfseries Reviewers:}
  \begin{itemize}
  \setlength\itemsep{0em}
    \item \href{https://github.com/Pending Reviewers}{@Pending
Reviewers}
    \end{itemize}
    \vspace{2mm}

  {\bfseries Submitted:} N/A\\
  {\bfseries Published:} N/A

  \vspace{2mm}
  {\bfseries License}\\
  Authors of papers retain copyright and release the work under a Creative Commons Attribution 4.0 International License (\href{http://creativecommons.org/licenses/by/4.0/}{\color{linky}{CC BY 4.0}}).

  
  \end{flushleft}
}

\hypertarget{abstract}{%
\section{Abstract}\label{abstract}}

We present a lightweight package to take care of the configuration and
organization of all your Python projects. Although \texttt{omni-fig} is
well suited for projects of virtually any size or complexity, it is
specifically designed for research projects where a small development
team doesn't want to waste time on boilerplate code or a bespoke user
interface. Nevertheless, the intuitive project structure encourages
developers to good design practices making it easier for collaborators
and prospective users to rapidly find and understand the core
contributions they need. The feature-rich configuration system
completely abstracts the implementation from the specification of
parameters, so the developer can focus entirely on creating new
functionality, while the user can quickly and precisely specify the
desired functionality by composing modular config files, through the
command line interface, or even in an interactive environment like
Jupyter. \texttt{omni-fig} transforms the project organization and
configuration from a bothersome distraction into a joy, thereby
improving clarity while also accelerating development.

\hypertarget{introduction}{%
\section{Introduction}\label{introduction}}

One particularly promising trend in computational research, especially
in machine learning, is that releasing the code associated with projects
is becoming increasingly common (Pineau et al., 2021). Not only is this
crucial for reproducibility (Guyon, n.d.), but this also fosters
interdisciplinary research and progress in industry (Aho et al., 2020).

However, the needs of developers differ significantly from those of
potential users. \emph{Developers} (and especially researchers) prefer
flexible, open-ended tools tailored for extensibility to prototype and
synthesize novel methods. Meanwhile, \emph{users} prefer tools that
automate details and provide a simple interface that is digestible and
directly applicable to some downstream problem. Even fellow researchers
exploring new methods want to quickly understand the essence of the
contribution rather than getting bogged down in some idiosyncrasies of a
project. For example, in machine learning, this has given rise to a
variety of development practices (Ebert et al., 2016; Treveil et al.,
2020) and AutoML packages (He et al., 2021) to make cutting edge methods
more highly accessible for a variety of real-world applications.
However, the opaque rigid interfaces and product-oriented focus cater
more to the end-user, thereby increasing friction in design and
synthesis for developers.

\hypertarget{statement-of-need}{%
\section{Statement of Need}\label{statement-of-need}}

Project organization and configuration is an important albeit
unglamorous part of every project. Neglecting this aspect can
significantly impede prototyping and development as well as making the
project less understandable and useable for any potential user or
collaborator. Consequently, quite a few packages already exist to help
with this task.

Perhaps the most similar to our proposed \texttt{omni-fig} is
\texttt{hydra} (Yadan, 2019), which is a popular configuration framework
that provides a feature-rich user interface, even including composing
config files to some extent. However, from the developer's perspective,
the project structure of \texttt{hydra} is more constrained and the
configuration system is built on top of \texttt{OmegaConf}
(\emph{OmegaConf - Flexible {P}ython Configuration System}, 2012),
making more advanced features such as automatic instantiation of objects
nonexistent or more cumbersome. Packages such as \texttt{OmegaConf} or
\texttt{pydantic} (Colvin, 2019), on the other hand, focus more on
integrations and performance but lack high-level features and a
user-friendly interface. Slightly less comparable are a variety of
packages designed for more specific applications with fixed project
structures, such as \texttt{dynaconf} (Rocha, 2018), \texttt{gin-config}
(Holtmann-Rice et al., 2018), and \texttt{confr} (Arro, 2022). Finally,
there are some built-in libraries that are commonly used for the
command-line interface and configuration, such as \texttt{argparse}
(\emph{Argparse - Parser for Command-Line Options, Arguments and
Sub-Commands}, n.d.) and \texttt{configparser} (\emph{Configparser -
Configuration File Parser}, n.d.). However, these provide minimal
features and scale poorly to more complex, evolving projects.

All too often, the trade-off between power and simplicity results in
software (particularly research projects) containing cutting-edge
features barely perceptible behind a limited or even nonexistent user
interface. Here, a good project configuration framework can bridge the
gap between developers and users. The user wants to select functionality
in a concise and intuitive way while still allowing fine-grained control
when needed. Meanwhile, the developer first and foremost wants the
configuration system to seamlessly provide the necessary parameters
without interfering with the broader design and implementation of the
functionality.

\hypertarget{summary}{%
\section{Summary}\label{summary}}

Here \texttt{omni-fig} strikes a favorable balance in that it was
designed from the ground up to cater specifically to the needs of both
the \emph{developer} and the \emph{user}.

The primary way the \emph{developer} interacts with \texttt{omni-fig} is
by registering any functionality as a \texttt{script} (for functions),
\texttt{component} (for classes), or \texttt{modifier} (for mixins),
creating config files as needed, and then accessing config parameters
using the config object at runtime. Once registered, the objects can be
accessed anywhere in the project through the config object, thereby
incentivizing the developer to register any functionality that can be
customized by configuration. Meanwhile, since config files can easily be
composed, the developer is incentivized to separate configuration in a
modular way. Finally, at runtime, the developer doesn't have to worry
about how the config parameters are specified (e.g., as a command line
argument or in a config file), but can simply access them through the
config object. This abstraction allows arbitrarily complex objects, even
including mix-ins added dynamically at runtime (see \texttt{modifier}),
to be instantiated automatically.

From the user's perspective, the modular config files and explicit
registration of top-level functionality greatly improve the transparency
of the project. For example, just running \texttt{fig\ -h} returns a
custom help message displaying all registered scripts in the project.
Then the high-level modularity of the config files allows the developer
to effortlessly create demos by composing existing config files to
showcase the key features of the project.

For more information, check out the
\href{https://omnifig.readthedocs.io/en/latest/}{documentation} which
includes an
\href{https://omnifig.readthedocs.io/en/latest/highlights.html}{overview}
of the key features with examples. We are also continuously working on
new features such as adding integrations and improving error messages.
In any case, contributions and feedback are always very welcome!

\hypertarget{acknowledgments}{%
\section{Acknowledgments}\label{acknowledgments}}

This work was supported by the German Federal Ministry of Education and
Research (BMBF): Tübingen AI Center, FKZ: 01IS18039B, and by the Machine
Learning Cluster of Excellence, EXC number 2064/1 -- Project number
390727645. The authors thank the International Max Planck Research
School for Intelligent Systems (IMPRS-IS) for supporting Felix Leeb. The
authors would also like to thank Vincent Berenz for his feedback and
suggestions, and Amanda Leeb for designing the omni-fig logo.

\hypertarget{references}{%
\section*{References}\label{references}}
\addcontentsline{toc}{section}{References}

\hypertarget{refs}{}
\begin{CSLReferences}{1}{0}
\leavevmode\hypertarget{ref-aho2020demystifying}{}%
Aho, T., Sievi-Korte, O., Kilamo, T., Yaman, S., \& Mikkonen, T. (2020).
Demystifying data science projects: A look on the people and process of
data science today. \emph{International Conference on Product-Focused
Software Process Improvement}, 153--167.
\url{https://doi.org/10.1007/978-3-030-64148-1_10}

\leavevmode\hypertarget{ref-argparse}{}%
\emph{Argparse - parser for command-line options, arguments and
sub-commands}. (n.d.).
\url{https://docs.python.org/3/library/argparse.html}

\leavevmode\hypertarget{ref-arro2022confr}{}%
Arro, M. (2022). \emph{Confr--a configuration system for machine
learning projects}.

\leavevmode\hypertarget{ref-pydantic}{}%
Colvin, S. (2019). \emph{Pydantic - data parsing and validation using
{P}ython type hints}. Github. \url{https://github.com/pydantic/pydantic}

\leavevmode\hypertarget{ref-configparser}{}%
\emph{Configparser - configuration file parser}. (n.d.).
\url{https://docs.python.org/3/library/configparser.html}

\leavevmode\hypertarget{ref-ebert2016devops}{}%
Ebert, C., Gallardo, G., Hernantes, J., \& Serrano, N. (2016). DevOps.
\emph{IEEE Software}, \emph{33}(3), 94--100.

\leavevmode\hypertarget{ref-guyonartificial}{}%
Guyon, I. (n.d.). \emph{Artificial intelligence for all}.

\leavevmode\hypertarget{ref-he2021automl}{}%
He, X., Zhao, K., \& Chu, X. (2021). AutoML: A survey of the
state-of-the-art. \emph{Knowledge-Based Systems}, \emph{212}, 106622.
\url{https://doi.org/10.1016/j.knosys.2020.106622}

\leavevmode\hypertarget{ref-ginconfig}{}%
Holtmann-Rice, D., Guadarrama, S., \& Silberman, N. (2018). \emph{Gin
config - provides a lightweight configuration framework for {P}ython}.
Github. \url{https://github.com/google/gin-config}

\leavevmode\hypertarget{ref-omegaconf}{}%
\emph{OmegaConf - flexible {P}ython configuration system}. (2012).
Github. \url{https://github.com/omry/omegaconf}

\leavevmode\hypertarget{ref-pineau2021improving}{}%
Pineau, J., Vincent-Lamarre, P., Sinha, K., Larivière, V., Beygelzimer,
A., d'Alché-Buc, F., Fox, E., \& Larochelle, H. (2021). Improving
reproducibility in machine learning research: A report from the NeurIPS
2019 reproducibility program. \emph{Journal of Machine Learning
Research}, \emph{22}.

\leavevmode\hypertarget{ref-dynaconf}{}%
Rocha, B. (2018). \emph{Dynaconf - configuration management for
{P}ython}. Github. \url{https://github.com/dynaconf/dynaconf}

\leavevmode\hypertarget{ref-treveil2020introducing}{}%
Treveil, M., Omont, N., Stenac, C., Lefevre, K., Phan, D., Zentici, J.,
Lavoillotte, A., Miyazaki, M., \& Heidmann, L. (2020). \emph{Introducing
MLOps}. O'Reilly Media.

\leavevmode\hypertarget{ref-hydra}{}%
Yadan, O. (2019). \emph{Hydra - a framework for elegantly configuring
complex applications}. Github.
\url{https://github.com/facebookresearch/hydra}

\end{CSLReferences}

\end{document}
